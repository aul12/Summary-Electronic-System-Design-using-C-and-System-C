\chapter{Electronic System Modelling}
Simulation performance is king.

\section{Design Abstraction Levels}
\begin{itemize}
    \item \textbf{ESL}: Electronic System Level: Major blocks of functionality and their connection
    \item \textbf{RTL}: Registers and combinatorial elements connected together
    \item \textbf{Gate Level}: Nets comprising AND/OR/NAND/NOR/XOR/DFF etc. gates
    \item \textbf{Transistor Level}: Nets connected PMOS/NMOS etc. transistors
    \item \textbf{Geometry Level}: Sets of mask with geometrical layout information
\end{itemize}

\section{Modelling Approaches}
Design Flows:
\begin{itemize}
    \item Top-Down: Start with overall design, refine blocks
    \item Bottom-Up: Design detailed block first, then combine
    \item Middle-Out: Combination of Top-Down, Bottom-Up.
\end{itemize}

Accuracy of the model:
\begin{itemize}
    \item Structural accuracy (pin accuracy)
    \item Timing accuracy (clock-cycle accuracy)
    \item Functional accuracy
    \item Data organization accuracy
    \item Communication protocol accuracy
\end{itemize}

\section{Models of Computation}
Fundamental to system-level design. Provides choices for accuracy/abstraction during design.

Need for different MOC that can interoperate with each other.

The MOC comprise:
\begin{itemize}
    \item Methods of communication
    \item Concept of time
    \item Rules for process activation
\end{itemize}

\section{Theory of Dataflow Models}
\begin{figure}[H]
    \centering
    \includegraphics[width=\textwidth]{dataflow.pdf}
\end{figure}
Alternative to the Von Neumann style:
\begin{itemize}
    \item Focus on movement of data
    \item Processes communicate through buffers
    \item Processes run simultaneously
    \item Sequence of data tikens read equals written (streaming)
    \item Scheduling done by a system
    \item Great for signal-processing applications
\end{itemize}

\subsection{Kahn Process Networks}
Unbound buffers, determinisitc, difficult to schedule (and obviously not possible to implement).

\subsection{Synchronous Dataflow}
Restricted KPN: Read and write fixed number of tokens, easier to schedule.

\section{Untimed Functional Modelling}
Determinisitic, untimed functional model: Behaviour is described as functionality that maps sequences of input tokens into sequences of output tokens.

With SystemC:
\begin{itemize}
    \item Processes communicate via FIFOs
    \item Processes use \lstinline{SC_THREAD} (Provides implicit synchronization)
    \item Finite size FIFOs with blocking read and blocking write
\end{itemize}

\section{Timing in Functional Modelling}
Introduce timing into Functional Models:
\begin{itemize}
    \item IO Delays
    \item Processing Delays
    \item Synchronization events 
\end{itemize}

With SystemC: Insert delays to the \lstinline{wait()} statements.

\section{Transaction-Level Modelling}
Compared to RTL and BL:
\begin{itemize}
    \item Give-up pin-accuracy
    \item Give-up communication/data accuracy
    \item High level model
    \item Details of communication are separated from details of implementation
    \item Emphasis on modelling functionality of data transfer but not actual implementation
    \item Can be untimed, timed or even bus-cycle accurate
\end{itemize}
Used when simulation speed is needed.

Benefits:
\begin{itemize}
    \item Easier to experiment with architectural changes
    \item Easier to design and use
    \item Available early in design process
    \item Ability to accurately (enough) model HW and SW
    \item High simulation speed (fast enough to validate SW before HW prototypes exist)
    \item Suppress uninteresting details
    \item Can serve as the "agreed-upon" interface between HW and Sw
    \item Can be a testing vehicle, for both HW and SW
\end{itemize}

With SystemC:
\begin{itemize}
    \item Channels are used as communication mechanism (for example \lstinline{sc_fifo})
    \item Transaction requests by calling functions
    \item Two Phase Synchronization Scheme
    \item Optional: priority handling
    \item Optional: Arbitration Schemes for Bus Access
\end{itemize}

For High-Performance
\begin{itemize}
    \item Model communication via transactions, not signals
    \item Give up pin accuracy
    \item Use C++ Datatypes
    \item Use pointers for efficiency
    \item Eliminate unnecessary context switches
    \item Don't use processes if you don't have to
    \item Use Methods, not Threads
\end{itemize}

\section{Parametrization}
Makes it possible to reuse components. Three possibilities:
\begin{itemize}
    \item Compile Time
    \item Elaboration Time
    \item Simulation Time
\end{itemize}
Do as soon as possible.

\section{Interface Design}
