\chapter{Introduction}
\section{What is System C}
\begin{itemize}
    \item Means to express concurrency
    \item Communication Mechanism
    \item Reactivity
    \item Concept of Time
    \item Event Driven Simulation Kernel
\end{itemize}

\section{Design Flow}
\begin{enumerate}
    \item Compilation
    \item Linking
    \item Execution
    \item Elaboration: SystemC Kernel connects and initializes design portions
    \item Simulation: SystenC Kernel works on event queue until there are no new events (or the kernel is stopped)
\end{enumerate}

\section{Hardware Data Types}
Sorted from fastest to slowest:
\begin{enumerate}
    \item C++ built in data types: \lstinline{int, unsigned, long, char, float, double}
    \item 1 to 64 bit integrals: \lstinline{sc_int<>, sc_uint<>}
    \item 2-valued boolean logic: \lstinline{sc_bit<>, sc_bv<>}
    \item 4-values logic: \lstinline{sc_logic<>, sc_lv<>}
    \item unlimited size numbers: \lstinline{sc_bigint<>, sc_biguint<>}
    \item fixed precision: \lstinline{sc_fixed<>, sc_ufixed<>}
\end{enumerate}

\section{Modules and Hierarchy}
\subsection{Modules}
Modules are containers for structure.

\subsection{Communication}
Communication is a triplet:
\begin{itemize}
    \item \textbf{Ports}: are doors into/out of a channel via a particular interface
    \item \textbf{Channels}: Implement a particular communication mechanism (e.g. FIFOs, stackes,...) 
    \item \textbf{Interfaces}: Show a modules communication methods (via abstract C++ class)
\end{itemize}

\subsection{Channels}
\subsubsection{Primitive channels}
Primitive channels are channels provided by System C, they have built in synchronization and are ready to use.

Examples:
\begin{itemize}
    \item \lstinline{sc_fifo<>}
    \item \lstinline{sc_mutex<>}
    \item \lstinline{sc_signal<>}
\end{itemize}

\subsubsection{Hierarchical channels}
Used to implement own channels for complexe communication or communication refinement.
Requires custom synchronization schemes.

\subsection{Concurrency via Processes}
A process is a basic unit of functionality, it runs concurrently to other processes.

A process is part of a module and needs to be registered with the simulation kernel.

There are two kinds of processes:
\begin{itemize}
    \item \lstinline{SC_METHOD}
    \item \lstinline{SC_THREAD}
\end{itemize}

\subsection{Events}
Events are the mechanism for reactivity, they control how processes are triggered and resumed.

Example: a signal has an event associated with each value change.

An event is notified by the owner:
\begin{itemize}
    \item Immediately: \lstinline{event.notify();}
    \item After a delta delay: \lstinline{event.notify(SC_ZERO_TIME);}
    \item After a certain time: \lstinline{event.notify(100, SC_NS);}
\end{itemize}

\section{Model of Time}
Time is represented by a integer (default: 64 bits, unsigned) value inside the simulation kernel.

Time units:
\begin{itemize}
    \item \lstinline{SC_FS}
    \item \lstinline{SC_PS}
    \item \lstinline{SC_NS}
    \item \lstinline{SC_US}
    \item \lstinline{SC_MS}
    \item \lstinline{SC_SEC}
\end{itemize}

\subsection{Clock module}
\lstinline{sc_clock clk("clk", 20, 0.5, 2, true);}
Arguments:
\begin{itemize}
    \item Name
    \item Period
    \item Duty-Cycle
    \item Phase Shift (as Time)
    \item First edge (\lstinline{true} is rising)
\end{itemize}

\section{Concept of Time for different Models}
\begin{figure}[H]
    \centering
    \includegraphics[width=.7\textwidth]{models.pdf}
\end{figure}

