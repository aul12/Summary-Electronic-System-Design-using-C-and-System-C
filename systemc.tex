\chapter{SystemC Language Elements}
\section{What is System C}
\begin{itemize}
    \item Modeling
        \begin{itemize}
            \item Concurrency
            \item Communication
            \item Reactivity
            \item Notion of Time
        \end{itemize}
    \item Simulation (Event-based simulation kernel)
    \item A C++ class library (like STL)
\end{itemize}

\section{Language Architecture}
\begin{figure}[H]
    \centering
    \includegraphics[width=\textwidth]{arch.pdf}
\end{figure}

\section{Design Flow}
\begin{enumerate}
    \item Compilation
    \item Linking
    \item Execution
    \item Elaboration: SystemC Kernel connects and initializes design portions
    \item Simulation: SystenC Kernel works on event queue until there are no new events (or the kernel is stopped)
\end{enumerate}

\section{Hardware Data Types}
Sorted from fastest to slowest:
\begin{enumerate}
    \item C++ built in data types: \lstinline{int, unsigned, long, char, float, double}
    \item 1 to 64 bit integrals: \lstinline{sc_int<>, sc_uint<>}
    \item 2-valued boolean logic: \lstinline{sc_bv<>}
    \item 4-values logic: \lstinline{sc_logic<>, sc_lv<>}
    \item unlimited size numbers: \lstinline{sc_bigint<>, sc_biguint<>}
    \item fixed precision: \lstinline{sc_fixed<>, sc_ufixed<>}
\end{enumerate}

\subsection{Native data types}
Fastest for simulation bugt inefficient in HW.

\subsection{1 to 64 bit integrals}
Paramatereisable width/precision, built on top of C++ integral types.
Native operators are overloaded.

\subsection{Arbitrary Precision Data Types}
Are used when 64 bits are not enough, e.g. in intermediate values.
Native operators are overloaded.

Beware of precision loss (because poor API design):
\begin{lstlisting}
    sc_uint<40> a,b;
    sc_biguint<80> w1 = a*b; // Calculation done with 64 bits

    // Better
    sc_biguint<80> w2 = a;
    w2 *= b;
\end{lstlisting}

\subsection{Bits and Bitvector}
Can have unlimited width. Only binary-logical operators are defined,
additionally \lstinline{length()}, \lstinline{print()}, \lstinline{lrotate()}, \lstinline{rrotate()}, \lstinline{reverse()}, \lstinline{operator[](int)} (\lstinline{[0]} is LSB) and \lstinline{operator()(int h, int l)} are defined.

\subsection{Logic Scalars and Logic Vectors}
Datatypes: \lstinline{sc_logic} and \lstinline{sc_lv}.

Four valued logic:
\begin{itemize}
    \item \lstinline{0}: low voltage
    \item \lstinline{1}: high voltage
    \item \lstinline{X}: undefined
    \item \lstinline{Z}: high impedance
\end{itemize}
With the and operation defined as:
\begin{table}[H]
    \centering
    \begin{tabular}{c|cccc}
        \toprule
        \& & 0 & 1 & X & Z \\
        \midrule
        0 & 0 & 0 & 0 & 0 \\
        1 & 0 & 1 & X & X \\
        X & 0 & X & X & X \\
        Z & 0 & X & X & X \\
        \bottomrule 
    \end{tabular}
\end{table}

Is used in busses with multiple drivers and/or multiple readers. 

The final, actual, value of the signal is then represented by the \lstinline{sc_signal_resolved} and \lstinline{sc_signal_rv} data type.

\subsection{Fixed-Point Data Types}
Floating-point not HW friendly. Fixed point logic solves the problem:
\begin{lstlisting}
    sc_fixed<WL, IWL, QM, OM, NB>
\end{lstlisting}
(or \lstinline{sc_ufixed<...>}) with:
\begin{enumerate}
    \item \lstinline{WL}: Word length (i.e. total length in bits)
    \item \lstinline{IWL}: Integer word length (i.e. position of "." relative to MSB
    \item \lstinline{QM}: Quantization mode
    \item \lstinline{OM}: Overflow mode (saturation or wrap-around
    \item \lstinline{NB}: Number of bit for overflow
\end{enumerate}

Faster but limited to 53 bits: \lstinline{sc_fixed_fast<>} and \lstinline{sc_ufixed_fast}.

\section{Modules}
Modules are containers for structure.

They contain:
\begin{itemize}
    \item Ports
    \item Processes
    \item Internal Data
    \item Sensitivity List
    \item Other modules (hierarchy)
\end{itemize}

\subsection{Syntax}
SystemC-Macro-Style:
\begin{lstlisting}
    SC_MODULE(Name) {
        ...
        SC_CTOR(Name) {

        }
    }
\end{lstlisting}

C++-Style:
\begin{lstlisting}
    class Name : public sc_module {
        SC_HAS_PROCESS(Name);   // Required
        Name(sc_module_name name, ...) : sc_module(name) {
            // CTor
        }
    }
\end{lstlisting}

The constructor can initialize ports: \lstinline{in1.initialize(0x17);}

\section{Communication}
Communication is a triplet.
\subsection{Ports}
Ports are the explicit (compare global variables) external interfaces of a module. They are members of the module class and are used to pass data to and from the module.

All ports have a datatype, and a direction:
\begin{itemize}
    \item \lstinline{sc_in<T>}
    \item \lstinline{sc_out<T>}
    \item \lstinline{sc_inout<T>}
\end{itemize}

\subsection{Interfaces}
Interfaces define a set of operations and data access. 
For example \lstinline{read()} and \lstinline{write()}.

Ports are bound to interfaces:
\begin{lstlisting}
    template <class T>
    class sc_in : public sc_port<sc_signal_in_if<T>, 1> {...
\end{lstlisting}

\subsection{Channels}
\begin{itemize}
    \item Channels (fully) implement one or more interfaces. 
        This makes it easy to replace them.
    \item Channels make concurrency work.
\end{itemize} 

\subsubsection{Primitive channels}
Primitive channels are channels provided by System C, they have built in synchronization and are ready to use.

Examples:
\begin{itemize}
    \item \lstinline{sc_fifo<>}
    \item \lstinline{sc_mutex<>}
    \item \lstinline{sc_signal<>}
\end{itemize}

Primitive channels are synchronized using the request-update-scheme (two phase synchronization):
\begin{lstlisting}
    class sc_prim_channel : ... {
        protected:
            void request_update(); // put into event queue
            virtual void update(); // called in 2nd phase
    }
\end{lstlisting}

\subsubsection{Hierarchical channels}
Used to implement own channels for complexe communication or communication refinement.
Requires custom synchronization schemes.

Hierarchical channels may have embedded:
\begin{itemize}
    \item Processes
    \item Instances of modules
    \item Channels
\end{itemize}

\section{Concurrency via Processes}
A process is a basic unit of functionality, it runs concurrently to other processes.

A process is part of a module and needs to be registered with the simulation kernel.

All processes follow the evaluate-update semantics of SystemC
\begin{enumerate}
    \item Invoke all processes, where events got notified
    \item Update channels with new values
\end{enumerate}

\subsection{Method Processes}
Entire body of member function is executed when an event is notigied
which the method is sensitive to. Execution is un-interrupted, in order and infinitely fast. Thus a method process has no implicit execution state.

Methods are good for combinatorial HW circuits, but the sensitivity is fixed at compile time.

\subsection{Thread Processes}
Threads have implicit state information. 
They suspend there execution using \lstinline{wait()}.

Threads have greater expressiveness which results in a greater cost in simulation performance. Threads are good for multi-cycle behaviour which require dynamic sensitivity.

\subsection{Possible Pitfalls}
\begin{itemize}
    \item Order of execution of the processes is determinisitic but unspecified
    \item Do not use shared resources between processes
    \item Do not call process directly, this will crash the simulation kernel
    \item No process should call another process directly
    \item Threads need to run into a \lstinline{wait()} at some point, beware of infinite loops.
\end{itemize}

\section{Events}
Events are the mechanism for reactivity, they control how processes are triggered and resumed.

Example: a signal has an event associated with each value change.

An event is notified by the owner:
\begin{itemize}
    \item Immediately: \lstinline{event.notify();}
    \item After a delta delay: \lstinline{event.notify(SC_ZERO_TIME);}
    \item After a certain time: \lstinline{event.notify(100, SC_NS);}
\end{itemize}

Processes can be made sensitive to events.

\subsection{Static sensitivity}
Special variable \lstinline{sensitive} of class \lstinline{sc_module} (either \lstinline{operator()(sc_event)} or \lstinline{operator<<(sc_event)}).

\subsection{Dynamic Sensitivity}
Using the \lstinline{wait()} statement for threads:
\begin{itemize}
    \item wait for an event: \lstinline{wait(e1);}
    \item all must be notified: \lstinline{wait(e1 & e2 & e3);}
    \item at least one must be notified: \lstinline{wait(e1 | e2 | e3);}
    \item wait for time: \lstinline{wait(214, SC_US);}
    \item wait for time of an event: \lstinline{wait(100, SC_NS, e1);}
\end{itemize}

Use \lstinline{next_trigger(time, time_unit, event)} for methods, this does not suspend the method.

\section{Model of Time}
Time is represented by a integer (default: 64 bits, unsigned) value inside the simulation kernel.

Time units:
\begin{itemize}
    \item \lstinline{SC_FS}
    \item \lstinline{SC_PS}
    \item \lstinline{SC_NS}
    \item \lstinline{SC_US}
    \item \lstinline{SC_MS}
    \item \lstinline{SC_SEC}
\end{itemize}

\subsection{Clock module}
\begin{lstlisting}
    sc_clock clk("clk", 20, 0.5, 2, true);
\end{lstlisting}

Arguments:
\begin{itemize}
    \item Name
    \item Period
    \item Duty-Cycle
    \item Phase Shift (as Time)
    \item First edge (\lstinline{true} is rising)
\end{itemize}

\subsection{Configuring Time}
\begin{lstlisting}
    sc_set_time_resolution(10, SC_PS);
\end{lstlisting}

Time is then rounded according to the resolution so
\lstinline{sc_time t(3.1415, SC_NS)} is equivalent to
\lstinline{sc_time t(3140, SC_PS)}.

\section{Concept of Time for Different Models}
\begin{figure}[H]
    \centering
    \includegraphics[width=.7\textwidth]{models.pdf}
\end{figure}

