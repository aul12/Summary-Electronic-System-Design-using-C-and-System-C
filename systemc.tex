\chapter{SystemC Language Elements}
\section{What is System C}
\begin{itemize}
    \item Modeling
        \begin{itemize}
            \item Concurrency
            \item Communication
            \item Reactivity
            \item Notion of Time
        \end{itemize}
    \item Simulation (Event-based simulation kernel)
    \item A C++ class library (like STL)
\end{itemize}

\section{Language Architecture}
\begin{figure}[H]
    \centering
    \includegraphics[width=\textwidth]{arch.pdf}
\end{figure}

\section{Design Flow}
\begin{enumerate}
    \item Compilation
    \item Linking
    \item Execution
    \item Elaboration: SystemC Kernel connects and initializes design portions
    \item Simulation: SystenC Kernel works on event queue until there are no new events (or the kernel is stopped)
\end{enumerate}

\section{Hardware Data Types}
Sorted from fastest to slowest:
\begin{enumerate}
    \item C++ built in data types: \lstinline{int, unsigned, long, char, float, double}
    \item 1 to 64 bit integrals: \lstinline{sc_int<>, sc_uint<>}
    \item 2-valued boolean logic: \lstinline{sc_bv<>}
    \item 4-values logic: \lstinline{sc_logic<>, sc_lv<>}
    \item unlimited size numbers: \lstinline{sc_bigint<>, sc_biguint<>}
    \item fixed precision: \lstinline{sc_fixed<>, sc_ufixed<>}
\end{enumerate}

\subsection{Native data types}
Fastest for simulation bugt inefficient in HW.

\subsection{1 to 64 bit integrals}
Paramatereisable width/precision, built on top of C++ integral types.
Native operators are overloaded.

\subsection{Arbitrary Precision Data Types}
Are used when 64 bits are not enough, e.g. in intermediate values.
Native operators are overloaded.

Beware of precision loss (because poor API design):
\begin{lstlisting}
    sc_uint<40> a,b;
    sc_biguint<80> w1 = a*b; // Calculation done with 64 bits

    // Better
    sc_biguint<80> w2 = a;
    w2 *= b;
\end{lstlisting}

\subsection{Bits and Bitvector}
Can have unlimited width. Only binary-logical operators are defined,
additionally \lstinline{length()}, \lstinline{print()}, \lstinline{lrotate()}, \lstinline{rrotate()}, \lstinline{reverse()}, \lstinline{operator[](int)} (\lstinline{[0]} is LSB) and \lstinline{operator()(int h, int l)} are defined.

\subsection{Logic Scalars and Logic Vectors}
Datatypes: \lstinline{sc_logic} and \lstinline{sc_lv}.

Four valued logic:
\begin{itemize}
    \item \lstinline{0}: low voltage
    \item \lstinline{1}: high voltage
    \item \lstinline{X}: undefined
    \item \lstinline{Z}: high impedance
\end{itemize}
With the and operation defined as:
\begin{table}[H]
    \centering
    \begin{tabular}{c|cccc}
        \toprule
        \& & 0 & 1 & X & Z \\
        \midrule
        0 & 0 & 0 & 0 & 0 \\
        1 & 0 & 1 & X & X \\
        X & 0 & X & X & X \\
        Z & 0 & X & X & X \\
        \bottomrule 
    \end{tabular}
\end{table}

Is used in busses with multiple drivers and/or multiple readers. 

The final, actual, value of the signal is then represented by the \lstinline{sc_signal_resolved} and \lstinline{sc_signal_rv} data type.

\subsection{Fixed-Point Data Types}
Floating-point not HW friendly. Fixed point logic solves the problem:
\begin{lstlisting}
    sc_fixed<WL, IWL, QM, OM, NB>
\end{lstlisting}
(or \lstinline{sc_ufixed<...>}) with:
\begin{enumerate}
    \item \lstinline{WL}: Word length (i.e. total length in bits)
    \item \lstinline{IWL}: Integer word length (i.e. position of "." relative to MSB
    \item \lstinline{QM}: Quantization mode
    \item \lstinline{OM}: Overflow mode (saturation or wrap-around
    \item \lstinline{NB}: Number of bit for overflow
\end{enumerate}

Faster but limited to 53 bits: \lstinline{sc_fixed_fast<>} and \lstinline{sc_ufixed_fast}.

\section{Modules and Hierarchy}
\subsection{Modules}
Modules are containers for structure.

They contain:
\begin{itemize}
    \item Ports
    \item Processes
    \item Internal Data
    \item Sensitivity List
    \item Other modules (hierarchy)
\end{itemize}

\subsection{Communication}
Communication is a triplet:
\begin{itemize}
    \item \textbf{Ports}: are doors into/out of a channel via a particular interface
    \item \textbf{Channels}: Implement a particular communication mechanism (e.g. FIFOs, stackes,...) 
    \item \textbf{Interfaces}: Show a modules communication methods (via abstract C++ class)
\end{itemize}

\subsection{Channels}
\subsubsection{Primitive channels}
Primitive channels are channels provided by System C, they have built in synchronization and are ready to use.

Examples:
\begin{itemize}
    \item \lstinline{sc_fifo<>}
    \item \lstinline{sc_mutex<>}
    \item \lstinline{sc_signal<>}
\end{itemize}

\subsubsection{Hierarchical channels}
Used to implement own channels for complexe communication or communication refinement.
Requires custom synchronization schemes.

\subsection{Concurrency via Processes}
A process is a basic unit of functionality, it runs concurrently to other processes.

A process is part of a module and needs to be registered with the simulation kernel.

There are two kinds of processes:
\begin{itemize}
    \item \lstinline{SC_METHOD}
    \item \lstinline{SC_THREAD}
\end{itemize}

\subsection{Events}
Events are the mechanism for reactivity, they control how processes are triggered and resumed.

Example: a signal has an event associated with each value change.

An event is notified by the owner:
\begin{itemize}
    \item Immediately: \lstinline{event.notify();}
    \item After a delta delay: \lstinline{event.notify(SC_ZERO_TIME);}
    \item After a certain time: \lstinline{event.notify(100, SC_NS);}
\end{itemize}

\section{Model of Time}
Time is represented by a integer (default: 64 bits, unsigned) value inside the simulation kernel.

Time units:
\begin{itemize}
    \item \lstinline{SC_FS}
    \item \lstinline{SC_PS}
    \item \lstinline{SC_NS}
    \item \lstinline{SC_US}
    \item \lstinline{SC_MS}
    \item \lstinline{SC_SEC}
\end{itemize}

\subsection{Clock module}
\begin{lstlisting}
    lstinline{sc_clock clk("clk", 20, 0.5, 2, true);}
\end{lstlisting}

Arguments:
\begin{itemize}
    \item Name
    \item Period
    \item Duty-Cycle
    \item Phase Shift (as Time)
    \item First edge (\lstinline{true} is rising)
\end{itemize}

\subsection{Configuring Time}
\begin{lstlisting}
    sc_set_time_resolution(10, SC_PS);
\end{lstlisting}

Time is then rounded according to the resolution so
\lstinline{sc_time t(3.1415, SC_NS)} is equivalent to
\lstinline{sc_time t(3140, SC_PS)}.

\section{Concept of Time for different Models}
\begin{figure}[H]
    \centering
    \includegraphics[width=.7\textwidth]{models.pdf}
\end{figure}

